\documentclass[11pt, openright]{book}

    % Cover Variables
    \newcommand{\ctitle}{ACTIVE FILTER TOPOLOGY}
    \newcommand{\cautor}{Lucas Lescure - Charlie Durand}
    \newcommand{\ctoptitle}{SECOND ORDER}

    % Header Variables
        \newcommand{\headRE}{\emph{\thepage}}
        \newcommand{\headLE}{\emph{\thesection. \rightmark}}
        \newcommand{\footRE}{}
        \newcommand{\footLE}{}

    % TOC Variables
        \newcommand{\toctitle}{Table of Content}
        \newcommand{\tocchapter}{Chapter}
        \newcommand{\toccount}{3}
  
    % Chapter Variables
        \newcommand{\chvar}{Chapter -}

\usepackage[a4paper, total={16cm, 22.125cm}]{geometry}

% Page Style
\usepackage[]{environ}
% Cover Page 
\usepackage{tikz}
\makeatletter
\def\parsecomma#1,#2\endparsecomma{\def\page@x{#1}\def\page@y{#2}}
\tikzdeclarecoordinatesystem{page}{
    \parsecomma#1\endparsecomma
    \pgfpointanchor{current page}{north east}
    % Save the upper right corner
    \pgf@xc=\pgf@x%
    \pgf@yc=\pgf@y%
    % save the lower left corner
    \pgfpointanchor{current page}{south west}
    \pgf@xb=\pgf@x%
    \pgf@yb=\pgf@y%
    % Transform to the correct placement
    \pgfmathparse{(\pgf@xc-\pgf@xb)/2.*\page@x+(\pgf@xc+\pgf@xb)/2.}
    \expandafter\pgf@x\expandafter=\pgfmathresult pt
    \pgfmathparse{(\pgf@yc-\pgf@yb)/2.*\page@y+(\pgf@yc+\pgf@yb)/2.}
    \expandafter\pgf@y\expandafter=\pgfmathresult pt
}
\makeatother


% Object formatting
\usepackage[12pt]{moresize}
\usepackage[]{anyfontsize}
\usepackage{titlesec}
\usepackage{import}
\usepackage{floatrow}
\usepackage{enumitem}
\usepackage{changepage}
\usepackage[normalem]{ulem}
\usepackage{array}
\newcommand{\ul}[1]{\underline{#1}}

\usepackage[]{chngcntr}
\usepackage{ifthen}
\ifthenelse{\figcountdepth > 1}
  {\counterwithin{figure}{section}\counterwithin{table}{section}}
  {}

\usepackage[format=plain, labelfont=it, textfont=it]{caption}
\makeatletter
\def\@makecaption#1#2{%
    \vskip\abovecaptionskip
    \sbox\@tempboxa{\textit{#1.} #2}

       
   

    \ifdim \wd\@tempboxa >\hsize
        #1. #2\par
    \else
        \global \@minipagefalse
        \hb@xt@\hsize{\hfil\box\@tempboxa\hfil}
    \fi
    \vskip\belowcaptionskip}
\makeatother

\DeclareCaptionFormat{underline}{\uline{#1#2#3}\par}

% Sections
\titleformat{\section}{\fontsize{16}{19.2}\bfseries}{\thesection.}{0.25em}{}
\titleformat{\subsection}{\fontsize{14}{16.8}\bfseries}{\tab\thesubsection.}{0.25em}{}
\titleformat{\subsubsection}{\fontsize{10}{12}}{\uline{\thesubsubsection)\enspace}}{0em}{\uline}





% Geometry

% Typewritting

\setlength{\parskip}{1em}
\setlength{\parindent}{0em}


\newenvironment{items}[3][0pt]
{\def\closesep{#3}
    \vspace{#2}
    \begin{itemize}
        \setlength{\itemsep}{#1}
        \setlength{\topsep}{0pt}
        \setlength{\partopsep}{0pt}}
        {\end{itemize}
    \vspace{\closesep}}

\newenvironment{enum}[3][0pt]
{\defclosesep{#3}
    \vspace{#2}
    \begin{enumerate}
        \setlength{\itemsep}{#1}
        \setlength{\topsep}{0pt}
        \setlength{\partopsep}{0pt}}
        {\end{enumerate}
    \vspace{\closesep}}

\newenvironment{eq}[2]
{\def\closesep{#2}
    \vspace{#1}
    \begin{align*}}
        {\end{align*}
    \vspace{\closesep}}

\newenvironment{lfeq}[2]
{\def\closesep{#2}
    \vspace{#1}
    \begin{flalign*}}
        {\end{flalign*}
    \vspace{\closesep}}
% List Formatting


\NewEnviron{dent}[1]{
    \vspace{-10pt}
    \begin{adjustwidth}{7mm}{}
        \uline{#1}\hspace{2mm}
        \BODY
    \end{adjustwidth}
    \vspace{-10pt}
}


\usepackage[framemethod=tikz]{mdframed}
\newcounter{count_theorem}[section]\setcounter{count_theorem}{0}
\newcommand{\thetheorem}{\arabic{count_theorem}}

\newcounter{count_exercise}[section]\setcounter{count_exercise}{0}
\newcommand{\theexercise}{\arabic{count_exercise}}


\newenvironment{theorem}[1][]{
    \refstepcounter{count_theorem}
    \mdfsetup{
        linecolor=red!30,
        innerbottommargin=10pt,
        linewidth=2pt,
        topline=false,
        bottomline=false,
        rightline=false,
        shadow=true,
        shadowsize=4.5pt,
        frametitlerule=false,
        apptotikzsetting={
                \tikzset{
                    mdfbackground/.append style={
                            left color=red!8,right color=red!3
                        }
                }
            }
    }
    \begin{mdframed}[]\relax
        \ifstrempty{#1}
        {\textbf{Theorem~\thetheorem.} }
        {\textbf{Theorem~\thetheorem.~#1} }
        }
        {\end{mdframed}\vspace{-10pt}
}

\newenvironment{note}{
    \mdfsetup{innertopmargin=5pt,
        linecolor=gray!30,
        linewidth=2pt,
        topline=false,
        bottomline=false,
        rightline=false,
        frametitleaboveskip=0pt,
        shadow=false,
        shadowsize=4pt,
        frametitlerule=false,
        apptotikzsetting={
                \tikzset{
                    mdfbackground/.append style={
                            left color=gray!8,right color=gray!3
                        }
                }
            }
    }
    \begin{mdframed}[]\relax
        \textbf{Note. }
        }
        {\end{mdframed}\vspace{-10pt}
}

\newenvironment{example}{
    \mdfsetup{innertopmargin=5pt,
        linecolor=green!30,
        linewidth=2pt,
        topline=false,
        bottomline=false,
        rightline=false,
        frametitleaboveskip=0pt,
        shadow=false,
        shadowsize=4pt,
        frametitlerule=false,
        apptotikzsetting={
                \tikzset{
                    mdfbackground/.append style={
                            left color=green!7,right color=green!2
                        },
                    mdfframetitlebackground/.append style={
                            left color=green!7,right color=green!2
                        }
                }
            }
    }
    \begin{mdframed}[]\relax
        \textbf{Example. }
        }
        {\end{mdframed}\vspace{-10pt}
}


\usetikzlibrary{calc,arrows}

\tikzset{
    excursus arrow/.style={%
            line width=2pt,
            draw=gray!40,
            rounded corners=2ex,
        },
    excursus head/.style={
            fill=white,
            font=\bfseries\sffamily,
            text=gray!80,
            anchor=base west,
        },
    excursus line/.style={%
            line width=2pt,
            draw=gray!40,
            rounded corners=2ex,
        }
}

\newenvironment{exercise}[1][]{%
    \refstepcounter{count_exercise}
    \mdfsetup{
        singleextra={
                \path let \p1=(P), \p2=(O) in (\x2,\y1) coordinate (Q);
                \path let \p1=(Q), \p2=(O) in (\x1,{(\y1-\y2)/2}) coordinate (M);
                \path [excursus line] ($(O)+(5em,0ex)$) -| (M) |- ($(Q)+(20em,0ex)$);
                \node [excursus head] at ($(Q)+(2.5em,-0.75pt)$) {\ifstrempty{#1}{Exercise \theexercise}{Exercise \theexercise:~#1}};},
        firstextra={
                \path let \p1=(P), \p2=(O) in (\x2,\y1) coordinate (Q);
                \path [excursus arrow,-to] (O) |- ($(Q)+(12em,0ex)$) .. controls +(0:16em) and +(185:6em) .. ++(23em,2ex);},
        middlelinewidth=2.5em,middlelinecolor=white,
        hidealllines=true,topline=true,
        innertopmargin=0.5ex,
        innerbottommargin=2.5ex,
        innerrightmargin=2pt,
        innerleftmargin=2ex,
        skipabove=0.87\baselineskip,
        skipbelow=0.62\baselineskip,
    }
    \begin{mdframed}[]\relax}
        {\end{mdframed}\vspace{-10pt}
}

% Functions and Data Plotting
\usepackage{subfig,wrapfig,adjustbox,multirow}


% Plotting Style
\usepackage{graphicx,pgfplots}
\usetikzlibrary{arrows}
\usetikzlibrary {patterns,patterns.meta}
\usepgfplotslibrary{fillbetween}
\pgfplotsset{compat=1.18}

\usepgfplotslibrary{units}
% Logarithmic Scale
\pgfplotsset{
    log x ticks with fixed point/.style={
            xticklabel={
                    \pgfkeys{/pgf/fpu=true}
                    \pgfmathparse{exp(\tick)}%
                    \pgfmathprintnumber[fixed relative, precision=3]{\pgfmathresult}
                    \pgfkeys{/pgf/fpu=false}
                }
        }
}


\input{/root/.config/latex-utils/common/math.tex}

% Headings  
\usepackage[Glenn]{fncychap}
\ChNumVar{\fontsize{40}{42}}
\ChTitleVar{\Large\sc}
\ChNameVar{\Large\sc}
\setlength\headheight{14.5pt}
\renewcommand\FmN[1]{\chvar}



\usepackage{fancyhdr}
\usepackage{ragged2e}

% Header & Footers
\renewcommand{\chaptermark}[1]{\markboth{#1}{#1}}
\renewcommand{\sectionmark}[1]{
    \markright{ #1}
}
\pagestyle{fancy}
\fancyhf{}
\fancyhead[LE,RO]{\headLE}
\fancyhead[RE,LO]{\headRE}
\fancyfoot[LE,RO]{\footLE}
\fancyfoot[RE,LO]{\footRE}
\renewcommand{\headrulewidth}{0.5pt}
\fancyheadoffset{1cm}

\fancypagestyle{plain}{%
    \fancyhf{} % clear all header and footer fields
    \fancyfoot[LE, RO]{\footLE}
    \renewcommand{\headrulewidth}{0pt}
    \renewcommand{\footrulewidth}{0pt}}


\fancypagestyle{nohead}{%
    \fancyhf{} % clear all header 
    \fancyfoot[LE, RO]{\footLE}
    \fancyfoot[LO, RE]{\footRE}}

    \fancypagestyle{head}{%
    \fancyhf{} % clear all header 
    \fancyhead[LE,RO]{\headLE}
\fancyhead[RE,LO]{\headRE}
\renewcommand{\headrulewidth}{0.5pt}
\fancyheadoffset{1cm}
    }


\fancypagestyle{bib}{%
    \fancyhf{} % clear all header and footer fields
    \fancyhead[CE, CO]{}
    \fancyfoot[LE, RO]{\footLE}
    \fancyfoot[LO, RE]{Bibliographie}}

% Table of Contents

\renewcommand*\thechapter{\arabic{chapter}} %Usually Roman
\renewcommand*\thesection{\arabic{section}}
\renewcommand*\thesubsubsection{\thesubsection.\alph{subsubsection}}
\makeatletter
\@removefromreset{section}{chapter}
\makeatother


% Table of Contents

\usepackage{titletoc}
\usepackage[linktoc=all]{hyperref}
\addto{\captionsenglish}{\renewcommand*{\contentsname}{\toctitle}}

\setcounter{secnumdepth}{3}
\setcounter{tocdepth}{\toccount}

\usepackage[subfigure]{tocloft}
\setlength\cftparskip{0pt}

\usepackage{etoolbox}
\makeatletter
\pretocmd{\chapter}{\addtocontents{toc}{\protect\addvspace{5\p@}}}{}{}
\pretocmd{\section}{\addtocontents{toc}{\protect\addvspace{-10\p@}}}{}{}
\pretocmd{\subsection}{\addtocontents{toc}{\protect\addvspace{1\p@}}}{}{}
\makeatother


% Chapter Style
\titlecontents{chapter}
[11em]
{\bigskip}
{\bfseries\textsc\tocchapter~\textsc\thecontentslabel : \textsc}
{\hspace*{-5.5em}\textbf}
{\titlerule*[1pc]{ }}[\smallskip]

% Section Style
\titlecontents{section}
[3em] % i
{\bigskip\bfseries}
{\fontsize{11}{13.2}\bfseries\uline{\thecontentslabel.\enspace}\uline}
{\hspace*{-4em}\textbf}
{\hspace{-2mm}\uline{\hspace*{\fill}}\hspace{-5pt}\contentspage}

% Subsection Style
\titlecontents{subsection}
[5em] % i
{\smallskip\bfseries}
{\fontsize{10}{12}\bfseries\thecontentslabel.\enspace}
{\hspace*{-4em}}
{\titlerule*[0.5pc]{.}\contentspage}



% Subsubsection Style
\titlecontents{subsubsection}
[7em] % i
{\smallskip}
{\fontsize{10}{12}\thecontentslabel)\enspace}
{\hspace*{-4em}}
{\titlerule*[0.5pc]{.}\contentspage}











    % figure support
    \usepackage{import}
    \usepackage{xifthen}
    \pdfminorversion=7
    \usepackage{pdfpages}
    \usepackage{transparent}
    \newcommand{\incfig}[1]{%
            \def\svgwidth{\columnwidth}
            \import{./figures/}{#1.pdf_tex}
    }

    \pdfsuppresswarningpagegroup=1

\begin{document}
% Spacing
\input{/root/.config/latex-utils/common/begin.tex}

% Cover
% Cover
\definecolor{ccolor1}{RGB}{236,145,143}
\definecolor{ccolor2}{RGB}{131,168,192}
\definecolor{ccolor3}{RGB}{182,227,150}
\definecolor{ccolor4}{RGB}{171,206,145}

\begin{titlepage}

    \newgeometry{top=1cm, width=21cm, bottom=1cm}

    \begin{tikzpicture}[remember picture,overlay,every node/.style={anchor=center}]
        \node[opacity =0.07, inner sep=0pt, anchor=east] at (current page.east){\includegraphics[width=0.5\paperwidth,height=\paperheight]{/home/archlinux/.config/latex-utils/logos/invert1.png}};

        %\node[opacity=0.15, inner sep=0pt, anchor=south west] at (current page.south west){\includegraphics[width=0.5\paperwidth,height=0.5\paperheight]{/home/archlinux/.config/latex-utils/logos/invert2.png}};

        \node[opacity=0.15,inner sep=0pt, anchor=north west] at (current page.north west){\includegraphics[width=0.5\paperwidth,height=0.5\paperheight]{/home/archlinux/.config/latex-utils/logos/invert3.png}};

        \node at (page cs:0,0.925) {\LARGE\bfseries\textsc{Telecom Saint-Étienne}};


        %\node[opacity=0.15, inner sep=0pt, anchor=south west] at (current page.south west){\includegraphics[width=0.5\paperwidth,height=0.5\paperheight]{/home/archlinux/.config/latex-utils/logos/invert2.png}};

        \node at (page cs:0,0.6) {\fontsize{28}{28.8}\textbf{\ctoptitle}};
        \node at (page cs:0,0.525) {\fontsize{28}{28.8}\textbf{\ctitle}};
        \draw (page cs:0.5,0.475) -- (page cs:-0.5,0.475);
        \node at (page cs:0,0.445) {\Large\textsc{}};
        \node at (page cs:0,0.4) {\Large\textsc{\cdate}};
        \node[anchor=east] at (page cs:-0.3,0.315) {\Large\textsc{Lucas Lescure: }};
        \node[anchor=west] at (page cs:-0.3,0.335) {\Large\textsc{lucas.lescure1@gmail.com}};
        \node[anchor=west] at (page cs:-0.3,0.295) {\large\textsc{+33 6 16 06 04 45 / +34 638 57 83 09}};
        
        \node[anchor=east]  at (page cs:-0.3,0.255) {\Large\textsc{Aubin Sionville: }};
        \node[anchor=west]  at (page cs:-0.3,0.255) {\Large\textsc{aubin.sionville@telecom-st-etienne.fr}};

        \node[anchor=east]  at (page cs:-0.3,0.215) {\Large\textsc{Tom Paillet: }};
        \node[anchor=west]  at (page cs:-0.3,0.215) {\Large\textsc{tom.paillet@telecom-st-etienne.fr}};

        \node at (page cs:0,0.9) {\includegraphics[height=1.5cm]{/home/archlinux/.config/latex-utils/logos/Logo.png}\hspace{11cm}\includegraphics[height=1.5cm]{/home/archlinux/.config/latex-utils/logos/UJM.png}};

    \end{tikzpicture}

    % Telecom Logo Big
    \begin{tikzpicture}[remember picture,overlay,every node/.style={anchor=south west}]
        \fill[thin, fill=ccolor1, opacity=1] (page cs: -1,-0.39) arc (124.5:-4.7:10cm) -- (page cs: -1,-1) -- (page cs: -1,-0.39);
        \fill[thin, fill=ccolor2, opacity=1] (page cs: -1,-0.14) arc (92.3:53.25:10cm) arc (85.9:124.5:10cm) -- (page cs: -1,-0.14);
        \fill[thin, fill=ccolor3, opacity=1] (page cs: -1,-0.005) arc (106:18.7:10cm) arc(49:85.9:10cm) arc (53.25:92.3:10cm) -- (page cs: -1,-0.005);
        \fill[thin, fill=ccolor4, opacity=1] (page cs: -0.06,-1) arc (-16.65:52.5:10cm) arc (85.9:49:10cm) arc (18.7:-34:10cm) -- (page cs: -0.06,-1);
        \fill[thin, fill=white, opacity=1] (page cs: -0.17,-1) -- (page cs: -0.17,-0.523) -- (page cs: -0.525,-0.374) -- (page cs: -1,-0.57) -- (page cs: -1,-1) -- (page cs: -0.925,-1) -- (page cs: -0.925,-0.833) -- (page cs: -0.515,-0.665) -- (page cs: -0.515,-1) -- (page cs: -0.17,-1);
    \end{tikzpicture}
\end{titlepage}


\newgeometry{left=2.5cm, width=16cm, bottom=2cm, top=2cm}

\tikz[remember picture, overlay] \node[opacity=0.15,inner sep=0pt, anchor=north east] at (current page.north east){\includegraphics[angle=-90,origin=c,width=0.5\paperheight,height=0.5\paperwidth]{/home/archlinux/.config/latex-utils/logos/invert3.png}};
\tikz[remember picture,overlay] \node[opacity=0.15,inner sep=0pt, anchor=south east] at (current page.south east){\includegraphics[angle=90,width=0.5\paperwidth,height=0.5\paperheight]{/home/archlinux/.config/latex-utils/logos/invert2.png}};

\tableofcontents

\newgeometry{left=2.5cm, width=16cm, bottom=2.5cm, top=2.5cm}



\vspace*{\fill}

\section{Abstract}

The goal of this report is to introduce ourselves to different second order active filter topologies. In this case we'll be studying the Sallen-Key and Multiple Feedback filters in an effort to understand and manipulate the effects of damping and resonance on these kinds of filters. 

\newpage

\section{Sallen-Key Low-Pass Filter}


\begin{figure}[ht]
  \centering
  \includegraphics{./Circuits/C1.jpeg}
  \caption{Sallen-Key Low-Pass Filter}
\end{figure}

\subsection{Study in sinusoidal regime}

\begin{wrapfigure}[14]{l}{0pt}
  \centering
  \begin{tikzpicture}
    \begin{axis}[
        width=0.45\textwidth, height=0.35\textwidth,
        xmode=log, log x ticks with fixed point,
        axis x line=bottom, axis y line=left,
        grid=both, ymax=17.5,xmax=100000,
        xlabel= Frequency ($H_z$), ylabel= Gain ($dB$)]
      
      \addplot[blue, smooth] table[x=Frequence, y=Gain, col sep=space] {./Bode/1.txt};
    \end{axis}
    \begin{axis}[
        width=0.45\textwidth, height=0.35\textwidth,
        xmode=log, log x ticks with fixed point,
        axis x line=bottom, axis y line=right,
        grid=none, ymax=87.5, ymin=-276,xmax=100000,
        xlabel= Frequency ($H_z$), ylabel= Phase]
      
      \addplot[red, smooth] table[x=Frequence, y=Phase, col sep=space] {./Bode/1.txt};
      \addplot[black, dashed, thick] coordinates {(1600,-276) (1600,73)};
      \addplot[black, dashed, thick] coordinates {(2500,-276) (2500,25)};
      \addplot[blue, dashed,very thick] coordinates {(2000,40) (90000,-276)};
      \addplot[red, dashed, very thick] coordinates {(1500,-5) (1500,-170)};
    \end{axis}
  \end{tikzpicture}
\end{wrapfigure}

Using the formulas provided in the preparation of the experiment we quickly calculate the damping coefficient $m=0.25$ as well as the constant gain factor $K=2.5$

Judging by the Bode diagram we can accurately measure a cut-off frequency at about $2500\ H_z$ as well as a resonance frequency at $1600\ H_z$ which maxes out at the curve's maximal gain $17.5\ dB$,  as well as a $-40\  dB/dec$ slope.	

On this non-unity-gain filter we notice that due to the damping factor being inferior to $1$ a peak of high gain manifests itself highlighting the resonance which can later be used for signal processing. 


\subsection{Study in impulsive regime}

\begin{wrapfigure}[8]{l}{0pt}
  \begin{tikzpicture}
    \begin{axis}[
        legend pos=south east,
        legend entries={$V_e$, $V_s$},
        axis lines = middle,
        ytick={5,2.5,-2.5,-5},
        yticklabels={5,2.5,-2.5,-5},
        xtick={-0.0075, -0.005, -0.0025, 0.0025, 0.005, 0.0075, 0.01, 0.0125},
        xticklabels={ , -5, , , 5, , 10, },
        xmin=-0.005, xmax=0.0075,
        ymin=-3, ymax=7.5, height=0.3\textwidth, width=0.5\textwidth,
        change x base,
        x SI prefix=milli,x unit=s,
        ylabel=\large{$V$},
        label style ={at={(ticklabel cs:1.1)}}]
      
      \addplot[blue, smooth] table[x=TIME, y=Ve, col sep=space]{./Data/passe bas1.txt};
      \addplot[orange, smooth] table[x=TIME, y=Vs, col sep=space]{./Data/passe bas1.txt};
    \end{axis}
  \end{tikzpicture}
  \caption{Chronograms of $V_s$ and $V_e$}
\end{wrapfigure}

To study the signal response we'll be choosing a frequency of $100\ H_z$ allowing us to clearly see the damped oscillations as the frequency goes below the resonance. 

This being done we can then measure the rise at about $124\ \mu s$, which can be considered slow . However the observed values correspond to our expectations since the damping coefficient remains lesser than $1$, responsible for the damped oscillations.

\newpage

\section{Multiple Feedback High-Pass Filter}

\begin{figure}[ht]
  \centering
  \includegraphics{./Circuits/C2.jpeg}
  \caption{Multiple Feedback High-Pass Filter}
\end{figure}

\subsection{Study in sinusoidal regime}


\begin{wrapfigure}[13]{l}{0pt}
  \centering
  \begin{tikzpicture}
    \begin{axis}[
        width=0.45\textwidth, height=0.35\textwidth,
        xmode=log, log x ticks with fixed point,
        axis x line=bottom, axis y line=left,
        grid=both, ymax=17.5,xmax=100000,
        xlabel= Frequency ($H_z$), ylabel= Gain ($dB$)]
      
      \addplot[blue, smooth] table[x=Frequence, y=Gain, col sep=space] {./Bode/2.txt};
    \end{axis}
    \begin{axis}[
        width=0.45\textwidth, height=0.35\textwidth,
        xmode=log, log x ticks with fixed point,
        axis x line=bottom, axis y line=right,
        grid=none, ymax=87.5, ymin=-330,xmax=100000,
        xlabel= Frequency ($H_z$), ylabel= Phase]
      
      \addplot[red, smooth] table[x=Frequence, y=Phase, col sep=space] {./Bode/2.txt};
      \addplot[black, dashed, thick] coordinates {(3000,-330) (3000,25)};
      \addplot[blue, dashed,very thick] coordinates {(2500,0) (60,-336)};
      \addplot[red, dashed, very thick] coordinates {(2600,-5) (2600,-170)};
    \end{axis}
  \end{tikzpicture}
\end{wrapfigure}

Using the formulas provided in the preparation of the experiment we quickly calculate the damping coefficient $m=1$ as well as the cut-off frequency at $f_0=2321\ Hz$

Judging by the Bode diagram we can accurately measure a cut-off frequency at about $2500\ H_z$ from which point rearwards the gain slope is at $40\ dB/dec$. All of which correspond to the expected values. Unlike last study, the damping coefficient is equal to 1 therefore the cut-off frequency is much clear and no resonance is expected. 

We can notice a second higher cut-off frequency at about $1300\ H_z$ this is due to the limitation of the operational amplifier which can no longer properly process the rapid input changes. 

\subsection{Study in impulsive regime}

\begin{wrapfigure}[11]{l}{0pt}
  \begin{tikzpicture}
    \begin{axis}[
        legend pos=south east,
        legend entries={$V_e$, $V_s$},
        axis lines = middle,
        ytick={2,1,-1,-2},
        yticklabels={2,1,-1,-2},
        xtick={0.005125, 0.00515, 0.005175 ,0.005200, 0.005225 , 0.005250},
        xticklabels={ , , 50, , 100, , 150, , 200 , ,},
        xmin=0.005125, xmax=0.00525,
        ymin=-2.5, ymax=2.5, height=0.3\textwidth, width=0.5\textwidth,
        change x base,
        x SI prefix=micro,x unit=s,
        ylabel=\large{$V$},
        label style ={at={(ticklabel cs:1.1)}}]
      
      \addplot[blue, smooth] table[x=TIME, y=Ve, col sep=space]{./Data/passe haut.txt};
      \addplot[orange, smooth] table[x=TIME, y=Vs, col sep=space]{./Data/passe haut.txt};
    \end{axis}
  \end{tikzpicture}
  \caption{Chronograms of $V_s$ and $V_e$}
\end{wrapfigure}

To study the signal response we'll be choosing a frequency of $10\ KH_z$ allowing us to clearly see the exponential decay as the frequency goes above cut-off. 

This being done we can then measure the rise at about $707\  ns$ .

The observed values correspond to our expectations since the damping coefficient remains equal to $1$, responsible for the exponential decay curve.


\end{document}













