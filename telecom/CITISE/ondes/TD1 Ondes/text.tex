\documentclass[11pt]{book}

% Cover Variables
	\newcommand{\ctitle}{Tittle}
	\newcommand{\cautor}{Autor}
% TOC Variables
	\newcommand{\toctitle}{Table of Content}
% Chapter Variables
	\newcommand{\chvar}{Chapter -}

\input{\string~/Projects/common/.style.tex}

\input{\string~/Projects/common/.math.tex}

\input{\string~/Projects/common/.header.tex}
	\pagestyle{fancy}
	\fancyhf{}
	\fancyhead[RE, LO]{Telecom St-Etienne \hfill CiTiSE 2}
	\fancyfoot[LE, RO]{\textbf{Page \thepage}}
	\fancyfoot[LO, RE]{\text{\rightmark}}
	
\input{\string~/Projects/common/.toc.tex}

% figure support
\usepackage{import}
\usepackage{xifthen}
\pdfminorversion=7
\usepackage{pdfpages}
\usepackage{transparent}
\newcommand{\incfig}[1]{%
	\def\svgwidth{\columnwidth}
	\import{./figures/}{#1.pdf_tex}
}

\pdfsuppresswarningpagegroup=1

\begin{document}
	% Spacing 
	\input{\string~/Projects/common/.begin.tex}
	
	\section{TD1 Ondes}

		\subsection{Exercise 1 : Ondes progressive}
		Montrer que $\ds{a(x,t)=f(x-ct)-f(x+ct)}$ est solution de l'équation d'Alembert : $\ds{\frac{\partial^2 a}{\partial x^2}-\frac{1}{c^2}\frac{\partial^2 a}{\partial t^2}=0}$.
		
		En considérant $\ds{ a(x,t)=f(x-ct)+g(x+ct)}$ comme solution de l'équation,
		 on applique les changement de variables suivant : $\ds{u=x-ct}$ et $\ds{v=x+ct}$

		Ainsi on développe: \\
		\centerline{$\ds{\Large\begin{cases}
			\frac{\partial a }{\partial x}=\frac{\partial f}{\partial u} \cdot \frac{\partial u}{\partial x} +\frac{\partial g}{\partial v} \cdot \frac{\partial v}{\partial x} =f'(u)+g'(v) \\
			\frac{\partial a}{\partial t} = \frac{\partial f}{\partial u} \cdot \frac{\partial u}{\partial t} + \frac{\partial g}{\partial v} \cdot \frac{\partial v}{\partial t} = -cf''(u)+cg''(v)  \end{cases}}$}
		Soit : \\
		\centerline{$\ds{\begin{cases}\begin{aligned}
					\frac{\partial^2 a}{\partial x^2}&=f''(u)\cdot \frac{\partial u}{\partial x} + g''(v)\cdot \frac{\partial v}{\partial x} = f''(u)+g''(v)\\
					\frac{\partial^2 a}{\partial t^2}&=-cf''(u)\cdot \frac{\partial u}{\partial t} +cg''(v)\cdot \frac{\partial v}{\partial t} =c^2\left( g''(v)+f''(u) \right) 
		\end{cases}\end{aligned}}$}

		On retrouve alors $\ds{f''(u)+g''(v)-\frac{1}{c^2}\cdot c^2\left( f''(u)+g''(v) \right)=0 }$ \\
		Ce qui est bien solution de l'équation d'Alembert. 

		\subsection{Exercise 2 : Relation de dispersion}

		En considérant une onde plane progressive monochromatique (OPPM) en notation complexe, retrouver la relation de dispersion caractéristique de l'équation d'Alembert $\ds{\omega =kx}$.




		



\end{document}
