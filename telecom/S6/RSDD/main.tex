\documentclass[11pt, openright]{book}

    % Cover Variables
    \newcommand{\ctoptitle}{}
    \newcommand{\ctitle}{Title}
    \newcommand{\cautor}{Author}
    \newcommand{\cdate}{day.month.year}
    \newcommand{\sectittle}{Second Title}


    % Header Variables
        \newcommand{\headRE}{Lucas Lescure }
        \newcommand{\headLE}{\emph{\rightmark}}
        \newcommand{\footRE}{Lucas Lescure $-$ \cdate}
        \newcommand{\footLE}{\emph{\thepage}}

    % TOC Variables
        \newcommand{\toctitle}{Table of Content}
        
        \newcommand{\tocchapter}{Chapter}
        \newcommand{\toccount}{3}
  
    % Chapter Variables
        \newcommand{\chvar}{Chapter -}

\usepackage[a4paper, total={16cm, 22.125cm}]{geometry}

% Page Style
\usepackage[]{environ}
% Cover Page 
\usepackage{tikz}
\makeatletter
\def\parsecomma#1,#2\endparsecomma{\def\page@x{#1}\def\page@y{#2}}
\tikzdeclarecoordinatesystem{page}{
    \parsecomma#1\endparsecomma
    \pgfpointanchor{current page}{north east}
    % Save the upper right corner
    \pgf@xc=\pgf@x%
    \pgf@yc=\pgf@y%
    % save the lower left corner
    \pgfpointanchor{current page}{south west}
    \pgf@xb=\pgf@x%
    \pgf@yb=\pgf@y%
    % Transform to the correct placement
    \pgfmathparse{(\pgf@xc-\pgf@xb)/2.*\page@x+(\pgf@xc+\pgf@xb)/2.}
    \expandafter\pgf@x\expandafter=\pgfmathresult pt
    \pgfmathparse{(\pgf@yc-\pgf@yb)/2.*\page@y+(\pgf@yc+\pgf@yb)/2.}
    \expandafter\pgf@y\expandafter=\pgfmathresult pt
}
\makeatother


% Object formatting
\usepackage[12pt]{moresize}
\usepackage[]{anyfontsize}
\usepackage{titlesec}
\usepackage{import}
\usepackage{floatrow}
\usepackage{enumitem}
\usepackage{changepage}
\usepackage[normalem]{ulem}
\usepackage{array}
\newcommand{\ul}[1]{\underline{#1}}

\usepackage[]{chngcntr}
\usepackage{ifthen}
\ifthenelse{\figcountdepth > 1}
  {\counterwithin{figure}{section}\counterwithin{table}{section}}
  {}

\usepackage[format=plain, labelfont=it, textfont=it]{caption}
\makeatletter
\def\@makecaption#1#2{%
    \vskip\abovecaptionskip
    \sbox\@tempboxa{\textit{#1.} #2}

       
   

    \ifdim \wd\@tempboxa >\hsize
        #1. #2\par
    \else
        \global \@minipagefalse
        \hb@xt@\hsize{\hfil\box\@tempboxa\hfil}
    \fi
    \vskip\belowcaptionskip}
\makeatother

\DeclareCaptionFormat{underline}{\uline{#1#2#3}\par}

% Sections
\titleformat{\section}{\fontsize{16}{19.2}\bfseries}{\thesection.}{0.25em}{}
\titleformat{\subsection}{\fontsize{14}{16.8}\bfseries}{\tab\thesubsection.}{0.25em}{}
\titleformat{\subsubsection}{\fontsize{10}{12}}{\uline{\thesubsubsection)\enspace}}{0em}{\uline}





% Geometry

% Typewritting

\setlength{\parskip}{1em}
\setlength{\parindent}{0em}


\newenvironment{items}[3][0pt]
{\def\closesep{#3}
    \vspace{#2}
    \begin{itemize}
        \setlength{\itemsep}{#1}
        \setlength{\topsep}{0pt}
        \setlength{\partopsep}{0pt}}
        {\end{itemize}
    \vspace{\closesep}}

\newenvironment{enum}[3][0pt]
{\defclosesep{#3}
    \vspace{#2}
    \begin{enumerate}
        \setlength{\itemsep}{#1}
        \setlength{\topsep}{0pt}
        \setlength{\partopsep}{0pt}}
        {\end{enumerate}
    \vspace{\closesep}}

\newenvironment{eq}[2]
{\def\closesep{#2}
    \vspace{#1}
    \begin{align*}}
        {\end{align*}
    \vspace{\closesep}}

\newenvironment{lfeq}[2]
{\def\closesep{#2}
    \vspace{#1}
    \begin{flalign*}}
        {\end{flalign*}
    \vspace{\closesep}}
% List Formatting


\NewEnviron{dent}[1]{
    \vspace{-10pt}
    \begin{adjustwidth}{7mm}{}
        \uline{#1}\hspace{2mm}
        \BODY
    \end{adjustwidth}
    \vspace{-10pt}
}


\usepackage[framemethod=tikz]{mdframed}
\newcounter{count_theorem}[section]\setcounter{count_theorem}{0}
\newcommand{\thetheorem}{\arabic{count_theorem}}

\newcounter{count_exercise}[section]\setcounter{count_exercise}{0}
\newcommand{\theexercise}{\arabic{count_exercise}}


\newenvironment{theorem}[1][]{
    \refstepcounter{count_theorem}
    \mdfsetup{
        linecolor=red!30,
        innerbottommargin=10pt,
        linewidth=2pt,
        topline=false,
        bottomline=false,
        rightline=false,
        shadow=true,
        shadowsize=4.5pt,
        frametitlerule=false,
        apptotikzsetting={
                \tikzset{
                    mdfbackground/.append style={
                            left color=red!8,right color=red!3
                        }
                }
            }
    }
    \begin{mdframed}[]\relax
        \ifstrempty{#1}
        {\textbf{Theorem~\thetheorem.} }
        {\textbf{Theorem~\thetheorem.~#1} }
        }
        {\end{mdframed}\vspace{-10pt}
}

\newenvironment{note}{
    \mdfsetup{innertopmargin=5pt,
        linecolor=gray!30,
        linewidth=2pt,
        topline=false,
        bottomline=false,
        rightline=false,
        frametitleaboveskip=0pt,
        shadow=false,
        shadowsize=4pt,
        frametitlerule=false,
        apptotikzsetting={
                \tikzset{
                    mdfbackground/.append style={
                            left color=gray!8,right color=gray!3
                        }
                }
            }
    }
    \begin{mdframed}[]\relax
        \textbf{Note. }
        }
        {\end{mdframed}\vspace{-10pt}
}

\newenvironment{example}{
    \mdfsetup{innertopmargin=5pt,
        linecolor=green!30,
        linewidth=2pt,
        topline=false,
        bottomline=false,
        rightline=false,
        frametitleaboveskip=0pt,
        shadow=false,
        shadowsize=4pt,
        frametitlerule=false,
        apptotikzsetting={
                \tikzset{
                    mdfbackground/.append style={
                            left color=green!7,right color=green!2
                        },
                    mdfframetitlebackground/.append style={
                            left color=green!7,right color=green!2
                        }
                }
            }
    }
    \begin{mdframed}[]\relax
        \textbf{Example. }
        }
        {\end{mdframed}\vspace{-10pt}
}


\usetikzlibrary{calc,arrows}

\tikzset{
    excursus arrow/.style={%
            line width=2pt,
            draw=gray!40,
            rounded corners=2ex,
        },
    excursus head/.style={
            fill=white,
            font=\bfseries\sffamily,
            text=gray!80,
            anchor=base west,
        },
    excursus line/.style={%
            line width=2pt,
            draw=gray!40,
            rounded corners=2ex,
        }
}

\newenvironment{exercise}[1][]{%
    \refstepcounter{count_exercise}
    \mdfsetup{
        singleextra={
                \path let \p1=(P), \p2=(O) in (\x2,\y1) coordinate (Q);
                \path let \p1=(Q), \p2=(O) in (\x1,{(\y1-\y2)/2}) coordinate (M);
                \path [excursus line] ($(O)+(5em,0ex)$) -| (M) |- ($(Q)+(20em,0ex)$);
                \node [excursus head] at ($(Q)+(2.5em,-0.75pt)$) {\ifstrempty{#1}{Exercise \theexercise}{Exercise \theexercise:~#1}};},
        firstextra={
                \path let \p1=(P), \p2=(O) in (\x2,\y1) coordinate (Q);
                \path [excursus arrow,-to] (O) |- ($(Q)+(12em,0ex)$) .. controls +(0:16em) and +(185:6em) .. ++(23em,2ex);},
        middlelinewidth=2.5em,middlelinecolor=white,
        hidealllines=true,topline=true,
        innertopmargin=0.5ex,
        innerbottommargin=2.5ex,
        innerrightmargin=2pt,
        innerleftmargin=2ex,
        skipabove=0.87\baselineskip,
        skipbelow=0.62\baselineskip,
    }
    \begin{mdframed}[]\relax}
        {\end{mdframed}\vspace{-10pt}
}

% Functions and Data Plotting
\usepackage{subfig,wrapfig,adjustbox,multirow}


% Plotting Style
\usepackage{graphicx,pgfplots}
\usetikzlibrary{arrows}
\usetikzlibrary {patterns,patterns.meta}
\usepgfplotslibrary{fillbetween}
\pgfplotsset{compat=1.18}

\usepgfplotslibrary{units}
% Logarithmic Scale
\pgfplotsset{
    log x ticks with fixed point/.style={
            xticklabel={
                    \pgfkeys{/pgf/fpu=true}
                    \pgfmathparse{exp(\tick)}%
                    \pgfmathprintnumber[fixed relative, precision=3]{\pgfmathresult}
                    \pgfkeys{/pgf/fpu=false}
                }
        }
}


\input{/home/archlinux/.config/latex-utils/common/math.tex}

% Headings  
\usepackage[Glenn]{fncychap}
\ChNumVar{\fontsize{40}{42}}
\ChTitleVar{\Large\sc}
\ChNameVar{\Large\sc}
\setlength\headheight{14.5pt}
\renewcommand\FmN[1]{\chvar}



\usepackage{fancyhdr}
\usepackage{ragged2e}

% Header & Footers
\renewcommand{\chaptermark}[1]{\markboth{#1}{#1}}
\renewcommand{\sectionmark}[1]{
    \markright{ #1}
}
\pagestyle{fancy}
\fancyhf{}
\fancyhead[LE,RO]{\headLE}
\fancyhead[RE,LO]{\headRE}
\fancyfoot[LE,RO]{\footLE}
\fancyfoot[RE,LO]{\footRE}
\renewcommand{\headrulewidth}{0.5pt}
\fancyheadoffset{1cm}

\fancypagestyle{plain}{%
    \fancyhf{} % clear all header and footer fields
    \fancyfoot[LE, RO]{\footLE}
    \renewcommand{\headrulewidth}{0pt}
    \renewcommand{\footrulewidth}{0pt}}


\fancypagestyle{nohead}{%
    \fancyhf{} % clear all header 
    \fancyfoot[LE, RO]{\footLE}
    \fancyfoot[LO, RE]{\footRE}}

    \fancypagestyle{head}{%
    \fancyhf{} % clear all header 
    \fancyhead[LE,RO]{\headLE}
\fancyhead[RE,LO]{\headRE}
\renewcommand{\headrulewidth}{0.5pt}
\fancyheadoffset{1cm}
    }


\fancypagestyle{bib}{%
    \fancyhf{} % clear all header and footer fields
    \fancyhead[CE, CO]{}
    \fancyfoot[LE, RO]{\footLE}
    \fancyfoot[LO, RE]{Bibliographie}}

% Table of Contents

\renewcommand*\thechapter{\arabic{chapter}} %Usually Roman
\renewcommand*\thesection{\arabic{section}}
\renewcommand*\thesubsubsection{\thesubsection.\alph{subsubsection}}
\makeatletter
\@removefromreset{section}{chapter}
\makeatother


% Table of Contents

\usepackage{titletoc}
\usepackage[linktoc=all]{hyperref}
\addto{\captionsenglish}{\renewcommand*{\contentsname}{\toctitle}}

\setcounter{secnumdepth}{3}
\setcounter{tocdepth}{\toccount}

\usepackage[subfigure]{tocloft}
\setlength\cftparskip{0pt}

\usepackage{etoolbox}
\makeatletter
\pretocmd{\chapter}{\addtocontents{toc}{\protect\addvspace{5\p@}}}{}{}
\pretocmd{\section}{\addtocontents{toc}{\protect\addvspace{-10\p@}}}{}{}
\pretocmd{\subsection}{\addtocontents{toc}{\protect\addvspace{1\p@}}}{}{}
\makeatother


% Chapter Style
\titlecontents{chapter}
[11em]
{\bigskip}
{\bfseries\textsc\tocchapter~\textsc\thecontentslabel : \textsc}
{\hspace*{-5.5em}\textbf}
{\titlerule*[1pc]{ }}[\smallskip]

% Section Style
\titlecontents{section}
[3em] % i
{\bigskip\bfseries}
{\fontsize{11}{13.2}\bfseries\uline{\thecontentslabel.\enspace}\uline}
{\hspace*{-4em}\textbf}
{\hspace{-2mm}\uline{\hspace*{\fill}}\hspace{-5pt}\contentspage}

% Subsection Style
\titlecontents{subsection}
[5em] % i
{\smallskip\bfseries}
{\fontsize{10}{12}\bfseries\thecontentslabel.\enspace}
{\hspace*{-4em}}
{\titlerule*[0.5pc]{.}\contentspage}



% Subsubsection Style
\titlecontents{subsubsection}
[7em] % i
{\smallskip}
{\fontsize{10}{12}\thecontentslabel)\enspace}
{\hspace*{-4em}}
{\titlerule*[0.5pc]{.}\contentspage}











    % figure support
    \usepackage{import}
    \usepackage{xifthen}
    \pdfminorversion=7
    \usepackage{pdfpages}
    \usepackage{transparent}
    \newcommand{\incfig}[1]{%
            \def\svgwidth{\columnwidth}
            \import{./figures/}{#1.pdf_tex}
    }

    \pdfsuppresswarningpagegroup=1


\begin{document}
% Spacing
\input{/home/archlinux/.config/latex-utils/common/begin.tex}

    \newgeometry{left=2.5cm, width=16cm, bottom=2.5cm, top=2.5cm}
    \thispagestyle{head}

     \section*{Retour d'experience et réflexion}

      \subsection*{Presentation du contexte}

      La banque alimentaire est une organisation qui lutte contre la faim et le gaspillage alimentaire en redistribuant des denrées alimentaires aux personnes en situation de précarité. Cette expérience s'inscrit dans le cadre d'une démarche personnelle de contribution à la société et de sensibilisation aux enjeux de la responsabilité sociétale et du développement durable.

      Il s'agit une association à but non lucratif, généralement régie par une structure fédérative. Chaque banque alimentaire est indépendante, mais elles sont toutes reliées au sein d'une fédération nationale, ce qui permet de coordonner leurs actions à l'échelle du pays

      Le rôle principal de la banque alimentaire est de récupérer des produits alimentaires invendus ou en surplus auprès des producteurs, des grandes surfaces, et d'autres donateurs. Ces produits sont ensuite triés, stockés et redistribués à des associations partenaires qui les distribuent directement aux bénéficiaires. Ces derniers sont principalement des personnes en situation de précarité, telles que les sans-abri, les familles à faible revenu, les étudiants en difficulté, et les personnes âgées isolées. Les associations partenaires, qui reçoivent les denrées de la banque alimentaire, jouent un rôle crucial dans la distribution de ces produits à ceux qui en ont le plus besoin.

       \subsubsection*{Objéctifs Développement Durable}

       En matière de responsabilité sociétale et de développement durable (RSDD) ses actions s'inscrivent dans quelques-un des 17 Objectifs de Développement Durable (ODD) définis par les Nations Unies :
        \begin{items}{-15pt}{-15pt}
           \item ODD 1 (Pas de pauvreté): En fournissant des denrées alimentaires aux plus démunis, la banque alimentaire contribue directement à réduire la pauvreté et à améliorer les conditions de vie des personnes en difficulté.
           \item ODD 2 (Faim zéro): En luttant contre le gaspillage alimentaire et en redistribuant des produits invendus, la banque alimentaire contribue à réduire la faim et à assurer la sécurité alimentaire des populations les plus vulnérables.
           \item ODD 12 (Consommation et production responsables): En récupérant des produits alimentaires invendus et en les redistribuant aux personnes en situation de précarité, la banque alimentaire promeut une consommation et une production responsables, en limitant le gaspillage et en favorisant la solidarité.
           \item ODD 17 (Partenariats pour la réalisation des objectifs): En travaillant en collaboration avec des producteurs, des grandes surfaces, des associations partenaires, et d'autres acteurs de la société civile, la banque alimentaire favorise les partenariats et la coopération pour atteindre ses objectifs de lutte contre la faim et le gaspillage alimentaire.
       \end{items}


     \subsection*{Présentation de l'activité}

     L'activité consistait à sensibiliser le public et à les encourager à acheter des sachets de nourriture pour contribuer à la banque alimentaire. Avec un partenaire, nous avons  discuté et sensibilisé les personnes à contribuer auprès de la banque alimentaire. L'objectif était d'informer les gens sur l'importance de la lutte contre la faim et le gaspillage alimentaire, et de les motiver à participer activement à cette cause en achetant des sachets de nourriture qui seraient ensuite redistribués aux personnes dans le besoin.

     Lors de cette activité, mes tâches spécifique étaient de communiqué avec les passants et les clients des commerces pour les informer sur les objectifs de notre opération et encourager les gens à acheter des sachets de nourriture en expliquant comment ces contributions seraient utilisées par la banque alimentaire. En collaboration avec mon partenaire, nous avons coordonné nos actions pour maximiser notre impact et atteindre nos objectifs communs.

     Au cours de cette activité, nous avons rencontré divers acteurs, notamment les clients des commerces, souvent des citoyens prêts à être sensibilisés aux enjeux sociétaux. Nous avons également interagi avec d'autres bénévoles de la banque alimentaire, engagés dans des activités similaires et partageant notre objectif de lutte contre la faim et le gaspillage alimentaire. Ces échanges ont été enrichissants et nous ont permis de renforcer notre engagement en faveur de cette cause.

     Notre action relevait de la responsabilité sociétale et du développement durable car elle visait à réduire la faim en augmentant les contributions alimentaires pour les personnes en difficulté, à sensibiliser le public aux enjeux du gaspillage alimentaire et à promouvoir des pratiques de consommation plus responsables. De plus, elle renforçait le tissu social en encourageant la solidarité et l'entraide communautaire, des valeurs essentielles pour une société durable et résiliente.

     Cette activité s'insère dans les problématiques réelles de la lutte contre la faim et la précarité alimentaire et la promotion d'une consommation responsable et solidaire. En sensibilisant le public à ces enjeux et en les encourageant à agir, nous avons contribué à apporter des solutions concrètes à des problèmes sociétaux et environnementaux majeurs.

     Les Objectifs de Développement Durable de l'ONU que notre action soutient sont :
      \begin{items}{-15pt}{-15pt}
         \item ODD 2 (Faim "zéro"): en augmentant les contributions alimentaires pour ceux qui en ont besoin 
         \item ODD 17 (Partenariats pour la réalisation des objectifs): en collaborant avec les commerces et les citoyens pour maximiser notre impact.
     \end{items}

     À court terme, notre action a permis d'augmenter immédiatement les stocks alimentaires de la banque alimentaire et de sensibiliser le public aux enjeux de la faim et du gaspillage alimentaire. À moyen terme, elle a renforcé les liens entre la banque alimentaire et les commerces locaux, et encouragé les contributions régulières des citoyens aux initiatives similaires. À long terme, notre action devrait contribuer à une réduction durable du gaspillage alimentaire grâce à une meilleure sensibilisation, à une amélioration continue des conditions de vie des personnes en situation de précarité grâce à un soutien alimentaire constant, et à l'établissement d'une culture de responsabilité sociétale et de solidarité au sein de la communauté.


        \subsection*{Retour réflexif d'experience}

        De cette opération, j'ai retiré une profonde satisfaction personnelle et un sentiment d'accomplissement en contribuant à une cause noble. 
    Cette expérience m'a permis de voir concrètement l'impact de nos actions sur les personnes en difficulté et de comprendre l'importance de l'engagement citoyen pour le bien-être collectif.

    En travaillant en équipe avec mon partenaire, nous avons pu mobiliser un grand nombre de personnes et augmenter significativement les contributions alimentaires à la banque alimentaire. Mon engagement et ma capacité à sensibiliser efficacement le public ont également permis de renforcer la visibilité de la cause et de susciter une prise de conscience plus large.

    Ce qui m'a étonné, c'est la générosité et l'enthousiasme des gens une fois informés des enjeux. Beaucoup de personnes étaient prêtes à contribuer dès qu'elles comprenaient l'importance de notre action. Cependant j'ai aussi été confronté à des réticences et des incompréhensions, ce qui m'a poussé à améliorer ma communication et à adapter mon discours en fonction des interlocuteurs.

    J'ai appris l'importance de la communication et de la sensibilisation dans la mobilisation des ressources et des personnes pour une cause sociale. J'ai également compris comment des actions concrètes, même à petite échelle, peuvent avoir un impact significatif lorsqu'elles sont bien organisées et soutenues par la communauté.

    Cette action a un impact sur mon écosystème externe en renforçant les liens sociaux et en sensibilisant la communauté à des pratiques de consommation plus responsables. En sensibilisant le public, nous avons également encouragé des comportements plus solidaires et responsables au sein de la communauté.


Cette action m'a fait réfléchir sur ma responsabilité en tant que citoyen et futur ingénieur. Elle m'a rappelé que chacun a un rôle à jouer dans la construction d'une société plus juste et durable. J'ai pris conscience de l'importance de l'engagement personnel dans des actions qui bénéficient à la collectivité et de la nécessité de promouvoir des pratiques durables dans toutes les sphères de la vie.

J'ai également réfléchi à la manière dont les ressources sont gérées et distribuées, et à l'impact de nos choix de consommation sur l'environnement et la société. Cette expérience m'a également sensibilisé à l'importance de la sensibilisation et de l'éducation pour promouvoir des comportements responsables.

Ma vision de la responsabilité sociétale et du développement durable est que chacun doit contribuer à la société de manière positive et proactive. Il est essentiel d'adopter des comportements qui minimisent notre impact environnemental tout en aidant ceux qui sont dans le besoin. En tant que futur ingénieur, je me vois jouer un rôle clé dans la création de solutions innovantes et durables qui peuvent répondre aux défis sociaux et environnementaux.

Cette action m'a fortement motivé à m'engager davantage dans des initiatives de responsabilité sociétale et de développement durable. Elle m'a montré que même des actions modestes peuvent avoir un impact significatif.

Professionnellement, je peux me concentrer sur le développement de technologies et de solutions qui favorisent la durabilité environnementale et sociale. En intégrant des principes de développement durable dans mes projets et en collaborant avec d'autres acteurs engagés, je peux contribuer à construire un futur plus durable et responsable.
\end{document}