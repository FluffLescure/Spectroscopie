\documentclass[11pt]{book}

% Cover Variables
	\newcommand{\ctitle}{Tittle}
	\newcommand{\cautor}{Autor}
% TOC Variables
	\newcommand{\toctitle}{Table of Content}
% Chapter Variables
	\newcommand{\chvar}{Cours Magistral}

\input{\string~/Projects/common/.style.tex}

\input{\string~/Projects/common/.math.tex}

\input{\string~/Projects/common/.header.tex}
	\pagestyle{fancy}
	\fancyhf{}
	\fancyhead[RE, LO]{\textbf{Cours Magistral}}
	\fancyfoot[LE, RO]{\textbf{Page \thepage}}
	\fancyfoot[LO, RE]{\text{\rightmark}}
	
\input{\string~/Projects/common/.toc.tex}

% figure support
\usepackage{import}
\usepackage{xifthen}
\pdfminorversion=7
\usepackage{pdfpages}
\usepackage{transparent}
\newcommand{\incfig}[1]{%
	\def\svgwidth{\columnwidth}
	\import{./figures/}{#1.pdf_tex}
}

\pdfsuppresswarningpagegroup=1

\begin{document}
	% Spacing 
	\input{\string~/Projects/common/.begin.tex}

	\section{CM 23/11/2022}

	\subsection{Développement limité }

 	$\ds{(1+x)^\alpha = a+\alpha x+\frac{\alpha (\alpha -a)}{2!}x^2+\ldots+\frac{\alpha (\alpha -1)\ldots(\alpha -n+1)}{n!}(x-\alpha )^2}$

	\subsection{Propriété de linéarité}

	Soient $\ds {f}$ et $\ds{g}$ admettant des développement limités au voisinage de 0 à l'ordre $\ds{n}$: \\
	\centerline{$\ds{f(x)=P_n(x)+o(x^n)}$} \\
	\centerline{$\ds{g(x)=Q_n(x)+o(x^n)}$}


	\subsection{Théorème de primitive}

	Soit $\ds{f}$ une fonction qui admet un développement limité en 0 à l'ordre $\ds{n}$ de partie régulière $\ds{P_n(x)}$:  \\
	\centerline{$\ds{f(x)=P_n(x)+o(x^n)}$}
	Si $\ds{f}$ admet un primitive $\ds{F}$ sur $\ds{l}$ alors $\ds{F}$ admet un développement limité en 0 à l'ordre $\ds{n+1}$: \\
	\centerline{$\ds{F(x)=F(0)+\int\limits_{0}^{x} P_n(t) \  d t +o(x^{n+1}) }$}

	\subsection{Produit de développement limitées}	

	Soit :
	$\ds{e^x=1+x+\frac{1}{2x^2+o\left(x^2  \right) }}$ et $\ds{\sin x= x+o(x^2)}$ 
	Alors on a : \\
	\centerline{$\ds{e^x \sin x= \left( 1+x+\frac{1}{2}x^2 \right)x  + \left( 1+x+\frac{1}{2}x^2 \right) \sin x +o\left(x^2  \right) }$}

	\subsection{Quotient de développement limités}

	Soient $\ds{f}$ et $\ds{g}$ deux fonctions admettant des développement limités à l'ordre $\ds{n}$ : \\
	\centerline{$\ds{f(x)=P_n(x)+o(x^{n})}$} \\
	\centerline{$\ds{g\left(x  \right) = Q_n(x)=o(x^n)}$}

	Si $\ds{Q_n(x)\neq 0}$ alors la fonction $\ds{\frac{f}{g}}$ admet un développement limité au voisinage de 0 à l'ordre $\ds{n}$.

	\begin{dent}{Exemple}
		Determiner le développement limité de $\ds{\frac{\sin x}{e^x}}$ en 0 à l'ordre 2.

		En passant par la division celon puissnace croissante on a : $\ds{\frac{x}{1+x+\frac{1}{2}x^2}=x-x^2}$ ainsi: \\
		\centerline{$\ds{\frac{\sin x}{e^x}=x-x^2}$}
	\end{dent}

\newpage

	\subsection{Composition de développement limités}

	Pour $\ds{f}$ et $\ds{g}$ admettant des développement limités en 0 à l'ordre $\ds{n}$. Si $\ds{u(0)=0}$ alors la fonction $\ds{f\circ u}$ admet un développement limité au voisinage de 0 à l'ordre $\ds{n}$ dont la partie régulière est obtenue en calculant la composée de $\ds{P_n(x)\circ Q_n(x)}$.

	\subsection{Propriété}

	Soit $\ds{f}$ une fonction admettant un développement limité en 0 à l'ordre $\ds{n}$. \\
	Alors $\ds{\forall \alpha \in \R^*}$ et $\ds{m \in \N^*}$ alors $\ds{g\mapsto f(\alpha x^{m})}$ possède un développement limité: \\
	\centerline{$\ds{g(x)=P_n(\alpha x^m)+o(x^{nm})}$}

	\section{Intégrales généralisés}

	\begin{dent}{Exemple}
		Calculer $\ds{\int\limits_{1}^{4} \frac{\left\lceil t \right\rceil }{t} \  d x }$ 

	\centerline{$\ds{\int\limits_{1}^{4} \frac{\left\lceil t \right\rceil }{t} \  d t = \int\limits_{1}^{2} \frac{\left\lceil t \right\rceil }{t} \  d t +\int\limits_{2}^{3} \frac{\left\lceil t \right\rceil }{t} \  d t +\int\limits_{3}^{4} \frac{\left\lceil t \right\rceil }{t} \  d t= \int\limits_{1}^{2} \frac{1}{t} \  d t +\int\limits_{2}^{3} \frac{2}{t} \  d t +\int\limits_{3}^{4} \frac{3}{t} \  d t  }$}

	\end{dent}
	 
	\subsection{Definition}

	Soit $\ds{ I}$ un intervalle de $\ds{\R}$. On dit qu'une fonction $\ds{f:l\to R}$ est localement Reinmann intégrable si et seulement si elle est Reinmann intégrable $\ds{\forall [a,b]\subset \R }$

	On dit que $\ds{\int\limits_{a}^{b} f(t) \  d t }$ converge si $\ds{\int\limits_{x}^{b} f(t) \  d t }$ converge ers une limite finie.
	

	\begin{dent}{Exemple :}
		$\ds{\int\limits_{0}^{1} \frac{1}{t^{\frac{1}{3}}} \  d t }$ 

		$\ds{f:t\mapsto \frac{1}{t^{\frac{1}{3}}}}$ est localement Reinmann intégrable sur $\ds{]0,1[}$\\
	\end{dent}
	

\end{document}
