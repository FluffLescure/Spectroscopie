\documentclass[11pt, openright]{book}

    % Cover Variables
    \newcommand{\ctoptitle}{}
    \newcommand{\ctitle}{Title}
    \newcommand{\cautor}{Author}
    \newcommand{\cdate}{day.month.year}
    \newcommand{\sectittle}{Second Title}


    % Header Variables
        \newcommand{\headRE}{Main Topic}
        \newcommand{\headLE}{\emph{\rightmark}}
        \newcommand{\footRE}{Lucas Lescure $-$ \cdate}
        \newcommand{\footLE}{\emph{\thepage}}

    % TOC Variables
        \newcommand{\toctitle}{Table of Content}
        
        \newcommand{\tocchapter}{Chapter}
        \newcommand{\toccount}{3}
  
    % Chapter Variables
        \newcommand{\chvar}{Chapter -}
        \newcommand{\figcountdepth}{1}


\usepackage[a4paper, total={16cm, 22.125cm}]{geometry}

% Page Style
\usepackage[]{environ}
% Cover Page 
\usepackage{tikz}
\makeatletter
\def\parsecomma#1,#2\endparsecomma{\def\page@x{#1}\def\page@y{#2}}
\tikzdeclarecoordinatesystem{page}{
    \parsecomma#1\endparsecomma
    \pgfpointanchor{current page}{north east}
    % Save the upper right corner
    \pgf@xc=\pgf@x%
    \pgf@yc=\pgf@y%
    % save the lower left corner
    \pgfpointanchor{current page}{south west}
    \pgf@xb=\pgf@x%
    \pgf@yb=\pgf@y%
    % Transform to the correct placement
    \pgfmathparse{(\pgf@xc-\pgf@xb)/2.*\page@x+(\pgf@xc+\pgf@xb)/2.}
    \expandafter\pgf@x\expandafter=\pgfmathresult pt
    \pgfmathparse{(\pgf@yc-\pgf@yb)/2.*\page@y+(\pgf@yc+\pgf@yb)/2.}
    \expandafter\pgf@y\expandafter=\pgfmathresult pt
}
\makeatother


% Object formatting
\usepackage[12pt]{moresize}
\usepackage[]{anyfontsize}
\usepackage{titlesec}
\usepackage{import}
\usepackage{floatrow}
\usepackage{enumitem}
\usepackage{changepage}
\usepackage[normalem]{ulem}
\usepackage{array}
\newcommand{\ul}[1]{\underline{#1}}

\usepackage[]{chngcntr}
\usepackage{ifthen}
\ifthenelse{\figcountdepth > 1}
  {\counterwithin{figure}{section}\counterwithin{table}{section}}
  {}

\usepackage[format=plain, labelfont=it, textfont=it]{caption}
\makeatletter
\def\@makecaption#1#2{%
    \vskip\abovecaptionskip
    \sbox\@tempboxa{\textit{#1.} #2}

       
   

    \ifdim \wd\@tempboxa >\hsize
        #1. #2\par
    \else
        \global \@minipagefalse
        \hb@xt@\hsize{\hfil\box\@tempboxa\hfil}
    \fi
    \vskip\belowcaptionskip}
\makeatother

\DeclareCaptionFormat{underline}{\uline{#1#2#3}\par}

% Sections
\titleformat{\section}{\fontsize{16}{19.2}\bfseries}{\thesection.}{0.25em}{}
\titleformat{\subsection}{\fontsize{14}{16.8}\bfseries}{\tab\thesubsection.}{0.25em}{}
\titleformat{\subsubsection}{\fontsize{10}{12}}{\uline{\thesubsubsection)\enspace}}{0em}{\uline}





% Geometry

% Typewritting

\setlength{\parskip}{1em}
\setlength{\parindent}{0em}


\newenvironment{items}[3][0pt]
{\def\closesep{#3}
    \vspace{#2}
    \begin{itemize}
        \setlength{\itemsep}{#1}
        \setlength{\topsep}{0pt}
        \setlength{\partopsep}{0pt}}
        {\end{itemize}
    \vspace{\closesep}}

\newenvironment{enum}[3][0pt]
{\defclosesep{#3}
    \vspace{#2}
    \begin{enumerate}
        \setlength{\itemsep}{#1}
        \setlength{\topsep}{0pt}
        \setlength{\partopsep}{0pt}}
        {\end{enumerate}
    \vspace{\closesep}}

\newenvironment{eq}[2]
{\def\closesep{#2}
    \vspace{#1}
    \begin{align*}}
        {\end{align*}
    \vspace{\closesep}}

\newenvironment{lfeq}[2]
{\def\closesep{#2}
    \vspace{#1}
    \begin{flalign*}}
        {\end{flalign*}
    \vspace{\closesep}}
% List Formatting


\NewEnviron{dent}[1]{
    \vspace{-10pt}
    \begin{adjustwidth}{7mm}{}
        \uline{#1}\hspace{2mm}
        \BODY
    \end{adjustwidth}
    \vspace{-10pt}
}


\usepackage[framemethod=tikz]{mdframed}
\newcounter{count_theorem}[section]\setcounter{count_theorem}{0}
\newcommand{\thetheorem}{\arabic{count_theorem}}

\newcounter{count_exercise}[section]\setcounter{count_exercise}{0}
\newcommand{\theexercise}{\arabic{count_exercise}}


\newenvironment{theorem}[1][]{
    \refstepcounter{count_theorem}
    \mdfsetup{
        linecolor=red!30,
        innerbottommargin=10pt,
        linewidth=2pt,
        topline=false,
        bottomline=false,
        rightline=false,
        shadow=true,
        shadowsize=4.5pt,
        frametitlerule=false,
        apptotikzsetting={
                \tikzset{
                    mdfbackground/.append style={
                            left color=red!8,right color=red!3
                        }
                }
            }
    }
    \begin{mdframed}[]\relax
        \ifstrempty{#1}
        {\textbf{Theorem~\thetheorem.} }
        {\textbf{Theorem~\thetheorem.~#1} }
        }
        {\end{mdframed}\vspace{-10pt}
}

\newenvironment{note}{
    \mdfsetup{innertopmargin=5pt,
        linecolor=gray!30,
        linewidth=2pt,
        topline=false,
        bottomline=false,
        rightline=false,
        frametitleaboveskip=0pt,
        shadow=false,
        shadowsize=4pt,
        frametitlerule=false,
        apptotikzsetting={
                \tikzset{
                    mdfbackground/.append style={
                            left color=gray!8,right color=gray!3
                        }
                }
            }
    }
    \begin{mdframed}[]\relax
        \textbf{Note. }
        }
        {\end{mdframed}\vspace{-10pt}
}

\newenvironment{example}{
    \mdfsetup{innertopmargin=5pt,
        linecolor=green!30,
        linewidth=2pt,
        topline=false,
        bottomline=false,
        rightline=false,
        frametitleaboveskip=0pt,
        shadow=false,
        shadowsize=4pt,
        frametitlerule=false,
        apptotikzsetting={
                \tikzset{
                    mdfbackground/.append style={
                            left color=green!7,right color=green!2
                        },
                    mdfframetitlebackground/.append style={
                            left color=green!7,right color=green!2
                        }
                }
            }
    }
    \begin{mdframed}[]\relax
        \textbf{Example. }
        }
        {\end{mdframed}\vspace{-10pt}
}


\usetikzlibrary{calc,arrows}

\tikzset{
    excursus arrow/.style={%
            line width=2pt,
            draw=gray!40,
            rounded corners=2ex,
        },
    excursus head/.style={
            fill=white,
            font=\bfseries\sffamily,
            text=gray!80,
            anchor=base west,
        },
    excursus line/.style={%
            line width=2pt,
            draw=gray!40,
            rounded corners=2ex,
        }
}

\newenvironment{exercise}[1][]{%
    \refstepcounter{count_exercise}
    \mdfsetup{
        singleextra={
                \path let \p1=(P), \p2=(O) in (\x2,\y1) coordinate (Q);
                \path let \p1=(Q), \p2=(O) in (\x1,{(\y1-\y2)/2}) coordinate (M);
                \path [excursus line] ($(O)+(5em,0ex)$) -| (M) |- ($(Q)+(20em,0ex)$);
                \node [excursus head] at ($(Q)+(2.5em,-0.75pt)$) {\ifstrempty{#1}{Exercise \theexercise}{Exercise \theexercise:~#1}};},
        firstextra={
                \path let \p1=(P), \p2=(O) in (\x2,\y1) coordinate (Q);
                \path [excursus arrow,-to] (O) |- ($(Q)+(12em,0ex)$) .. controls +(0:16em) and +(185:6em) .. ++(23em,2ex);},
        middlelinewidth=2.5em,middlelinecolor=white,
        hidealllines=true,topline=true,
        innertopmargin=0.5ex,
        innerbottommargin=2.5ex,
        innerrightmargin=2pt,
        innerleftmargin=2ex,
        skipabove=0.87\baselineskip,
        skipbelow=0.62\baselineskip,
    }
    \begin{mdframed}[]\relax}
        {\end{mdframed}\vspace{-10pt}
}

% Functions and Data Plotting
\usepackage{subfig,wrapfig,adjustbox,multirow}


% Plotting Style
\usepackage{graphicx,pgfplots}
\usetikzlibrary{arrows}
\usetikzlibrary {patterns,patterns.meta}
\usepgfplotslibrary{fillbetween}
\pgfplotsset{compat=1.18}

\usepgfplotslibrary{units}
% Logarithmic Scale
\pgfplotsset{
    log x ticks with fixed point/.style={
            xticklabel={
                    \pgfkeys{/pgf/fpu=true}
                    \pgfmathparse{exp(\tick)}%
                    \pgfmathprintnumber[fixed relative, precision=3]{\pgfmathresult}
                    \pgfkeys{/pgf/fpu=false}
                }
        }
}


\input{~/.config/latex-utils/common/math.tex}

% Headings  
\usepackage[Glenn]{fncychap}
\ChNumVar{\fontsize{40}{42}}
\ChTitleVar{\Large\sc}
\ChNameVar{\Large\sc}
\setlength\headheight{14.5pt}
\renewcommand\FmN[1]{\chvar}



\usepackage{fancyhdr}
\usepackage{ragged2e}

% Header & Footers
\renewcommand{\chaptermark}[1]{\markboth{#1}{#1}}
\renewcommand{\sectionmark}[1]{
    \markright{ #1}
}
\pagestyle{fancy}
\fancyhf{}
\fancyhead[LE,RO]{\headLE}
\fancyhead[RE,LO]{\headRE}
\fancyfoot[LE,RO]{\footLE}
\fancyfoot[RE,LO]{\footRE}
\renewcommand{\headrulewidth}{0.5pt}
\fancyheadoffset{1cm}

\fancypagestyle{plain}{%
    \fancyhf{} % clear all header and footer fields
    \fancyfoot[LE, RO]{\footLE}
    \renewcommand{\headrulewidth}{0pt}
    \renewcommand{\footrulewidth}{0pt}}


\fancypagestyle{nohead}{%
    \fancyhf{} % clear all header 
    \fancyfoot[LE, RO]{\footLE}
    \fancyfoot[LO, RE]{\footRE}}

    \fancypagestyle{head}{%
    \fancyhf{} % clear all header 
    \fancyhead[LE,RO]{\headLE}
\fancyhead[RE,LO]{\headRE}
\renewcommand{\headrulewidth}{0.5pt}
\fancyheadoffset{1cm}
    }


\fancypagestyle{bib}{%
    \fancyhf{} % clear all header and footer fields
    \fancyhead[CE, CO]{}
    \fancyfoot[LE, RO]{\footLE}
    \fancyfoot[LO, RE]{Bibliographie}}

% Table of Contents

\renewcommand*\thechapter{\arabic{chapter}} %Usually Roman
\renewcommand*\thesection{\arabic{section}}
\renewcommand*\thesubsubsection{\thesubsection.\alph{subsubsection}}
\makeatletter
\@removefromreset{section}{chapter}
\makeatother


% Table of Contents

\usepackage{titletoc}
\usepackage[linktoc=all]{hyperref}
\addto{\captionsenglish}{\renewcommand*{\contentsname}{\toctitle}}

\setcounter{secnumdepth}{3}
\setcounter{tocdepth}{\toccount}

\usepackage[subfigure]{tocloft}
\setlength\cftparskip{0pt}

\usepackage{etoolbox}
\makeatletter
\pretocmd{\chapter}{\addtocontents{toc}{\protect\addvspace{5\p@}}}{}{}
\pretocmd{\section}{\addtocontents{toc}{\protect\addvspace{-10\p@}}}{}{}
\pretocmd{\subsection}{\addtocontents{toc}{\protect\addvspace{1\p@}}}{}{}
\makeatother


% Chapter Style
\titlecontents{chapter}
[11em]
{\bigskip}
{\bfseries\textsc\tocchapter~\textsc\thecontentslabel : \textsc}
{\hspace*{-5.5em}\textbf}
{\titlerule*[1pc]{ }}[\smallskip]

% Section Style
\titlecontents{section}
[3em] % i
{\bigskip\bfseries}
{\fontsize{11}{13.2}\bfseries\uline{\thecontentslabel.\enspace}\uline}
{\hspace*{-4em}\textbf}
{\hspace{-2mm}\uline{\hspace*{\fill}}\hspace{-5pt}\contentspage}

% Subsection Style
\titlecontents{subsection}
[5em] % i
{\smallskip\bfseries}
{\fontsize{10}{12}\bfseries\thecontentslabel.\enspace}
{\hspace*{-4em}}
{\titlerule*[0.5pc]{.}\contentspage}



% Subsubsection Style
\titlecontents{subsubsection}
[7em] % i
{\smallskip}
{\fontsize{10}{12}\thecontentslabel)\enspace}
{\hspace*{-4em}}
{\titlerule*[0.5pc]{.}\contentspage}











    % figure support
    \usepackage{import}
    \usepackage{xifthen}
    %\pdfminorversion=7
    \usepackage{pdfpages}
    \usepackage{transparent}
    \newcommand{\incfig}[1]{%
            \def\svgwidth{\columnwidth}
            \import{./figures/}{#1.pdf_tex}
    }

    %\pdfsuppresswarningpagegroup=1
    


\begin{document}
% Spacing
\input{~/.config/latex-utils/common/begin.tex}

% Cover
%% Cover
\definecolor{ccolor1}{RGB}{236,145,143}
\definecolor{ccolor2}{RGB}{131,168,192}
\definecolor{ccolor3}{RGB}{182,227,150}
\definecolor{ccolor4}{RGB}{171,206,145}

\begin{titlepage}

    \newgeometry{top=1cm, width=21cm, bottom=1cm}

    \begin{tikzpicture}[remember picture,overlay,every node/.style={anchor=center}]
        \node[opacity =0.07, inner sep=0pt, anchor=east] at (current page.east){\includegraphics[width=0.5\paperwidth,height=\paperheight]{/home/archlinux/.config/latex-utils/logos/invert1.png}};

        %\node[opacity=0.15, inner sep=0pt, anchor=south west] at (current page.south west){\includegraphics[width=0.5\paperwidth,height=0.5\paperheight]{/home/archlinux/.config/latex-utils/logos/invert2.png}};

        \node[opacity=0.15,inner sep=0pt, anchor=north west] at (current page.north west){\includegraphics[width=0.5\paperwidth,height=0.5\paperheight]{/home/archlinux/.config/latex-utils/logos/invert3.png}};

        \node at (page cs:0,0.925) {\LARGE\bfseries\textsc{Telecom Saint-Étienne}};


        %\node[opacity=0.15, inner sep=0pt, anchor=south west] at (current page.south west){\includegraphics[width=0.5\paperwidth,height=0.5\paperheight]{/home/archlinux/.config/latex-utils/logos/invert2.png}};

        \node at (page cs:0,0.6) {\fontsize{28}{28.8}\textbf{\ctoptitle}};
        \node at (page cs:0,0.525) {\fontsize{28}{28.8}\textbf{\ctitle}};
        \draw (page cs:0.5,0.475) -- (page cs:-0.5,0.475);
        \node at (page cs:0,0.445) {\Large\textsc{}};
        \node at (page cs:0,0.4) {\Large\textsc{\cdate}};
        \node[anchor=east] at (page cs:-0.3,0.315) {\Large\textsc{Lucas Lescure: }};
        \node[anchor=west] at (page cs:-0.3,0.335) {\Large\textsc{lucas.lescure1@gmail.com}};
        \node[anchor=west] at (page cs:-0.3,0.295) {\large\textsc{+33 6 16 06 04 45 / +34 638 57 83 09}};
        
        \node[anchor=east]  at (page cs:-0.3,0.255) {\Large\textsc{Aubin Sionville: }};
        \node[anchor=west]  at (page cs:-0.3,0.255) {\Large\textsc{aubin.sionville@telecom-st-etienne.fr}};

        \node[anchor=east]  at (page cs:-0.3,0.215) {\Large\textsc{Tom Paillet: }};
        \node[anchor=west]  at (page cs:-0.3,0.215) {\Large\textsc{tom.paillet@telecom-st-etienne.fr}};

        \node at (page cs:0,0.9) {\includegraphics[height=1.5cm]{/home/archlinux/.config/latex-utils/logos/Logo.png}\hspace{11cm}\includegraphics[height=1.5cm]{/home/archlinux/.config/latex-utils/logos/UJM.png}};

    \end{tikzpicture}

    % Telecom Logo Big
    \begin{tikzpicture}[remember picture,overlay,every node/.style={anchor=south west}]
        \fill[thin, fill=ccolor1, opacity=1] (page cs: -1,-0.39) arc (124.5:-4.7:10cm) -- (page cs: -1,-1) -- (page cs: -1,-0.39);
        \fill[thin, fill=ccolor2, opacity=1] (page cs: -1,-0.14) arc (92.3:53.25:10cm) arc (85.9:124.5:10cm) -- (page cs: -1,-0.14);
        \fill[thin, fill=ccolor3, opacity=1] (page cs: -1,-0.005) arc (106:18.7:10cm) arc(49:85.9:10cm) arc (53.25:92.3:10cm) -- (page cs: -1,-0.005);
        \fill[thin, fill=ccolor4, opacity=1] (page cs: -0.06,-1) arc (-16.65:52.5:10cm) arc (85.9:49:10cm) arc (18.7:-34:10cm) -- (page cs: -0.06,-1);
        \fill[thin, fill=white, opacity=1] (page cs: -0.17,-1) -- (page cs: -0.17,-0.523) -- (page cs: -0.525,-0.374) -- (page cs: -1,-0.57) -- (page cs: -1,-1) -- (page cs: -0.925,-1) -- (page cs: -0.925,-0.833) -- (page cs: -0.515,-0.665) -- (page cs: -0.515,-1) -- (page cs: -0.17,-1);
    \end{tikzpicture}
\end{titlepage}


\newgeometry{left=2.5cm, width=16cm, bottom=2cm, top=2cm}

\tikz[remember picture, overlay] \node[opacity=0.15,inner sep=0pt, anchor=north east] at (current page.north east){\includegraphics[angle=-90,origin=c,width=0.5\paperheight,height=0.5\paperwidth]{/home/archlinux/.config/latex-utils/logos/invert3.png}};
\tikz[remember picture,overlay] \node[opacity=0.15,inner sep=0pt, anchor=south east] at (current page.south east){\includegraphics[angle=90,width=0.5\paperwidth,height=0.5\paperheight]{/home/archlinux/.config/latex-utils/logos/invert2.png}};

\tableofcontents

\newgeometry{left=2.5cm, width=16cm, bottom=2.5cm, top=2.5cm}


    
\pagestyle{plain}
    \newpage

    \section*{Fabrication process of long period gratings in SMF-28 Corning fiber using point-by-point method}
    This is a short report on the fabrication process of long period gratings(LPG) using a 510nm femto-second laser and the point-by-point method on a SMF-28 Corning fiber. Other fabrication methods such as the line-by-line method have not been studied, but remain an  interesting and feasible alternative. 

    To control the inscription process of the LPG, a custom-made Direct Machine Control(DMC) recipe was used which is based on existing recipes for fiber Bragg gratings(FBG). Modification were made to the recipe to account for the large periods of the LPGs. The recipe mainly takes advantage of the \texttt{gating} array to apply a periodic rectangular gating function to the grating. Thereby converting an FBG inscription into an LPG inscription. A detailed explanation of the recipe will later be provided in this report.

    This report is not a a perfect nor detailed guide to the fabrication of LPGs, but rather a collection of notes and observations made while attempting to find a reliable and consistent fabrication process. 

        \subsection*{Alignment of the fiber}

        The alignment of the fiber is crucial to the fabrication of the LPG. During attempts at fabricating LPGs, a perfect alignment was never truly obtained, however a near perfect alignment deviating by no more than $\pm1\mu m$ did yield the best results. Other attempts with larger deviations resulted in seemingly random and inconsistent LPGs.
        
        During testing large deviation($\pm 3\mu m$) were commonplace. Despite efforts to reduce the deviation by pulling the fiber through the fiber holder, the deviation still remained. The source of these deviations remains uncertain. A number of possible factors come to mind, fabrication tolerances or fiber deformation caused by strain. I suspect surface tension of the refractive index liquid can also cause slight deviations, since large bubbles present in the liquid and close to the fiber seemed to correlate with regions where large deviations were observed while aligning the fiber. To reduce this, the refractive index liquid was applied carefully along the fiber making sure that none of the support fiber strands touched the liquid and making sure no big bubbles were created.

        Another source of deviations where due to the glass slide under the fiber being to low. This caused the fiber to bend a fair amount over the Z-axis, especially near the extremities of the glass cover. To mitigate this, two pieces of paper were placed under the glass slide to rise it up [\hyperref[fig:setup]{Figure \ref{fig:setup}}] . This helped to reduce the deviations of the fiber during the alignment process.
        
         \begin{figure}[ht!]
            \centering
            \includegraphics[width=0.5\textwidth]{./includes/setup2.jpg}
            \caption{Fiber holder setup with paper under the glass slide}
            \label{fig:setup}
        \end{figure}

\newpage

        After setting the fiber properly in the fiber holder, a significant amount of pull was applied to the fiber to straighten it as best as possible. To avoid the fiber sliding through the clamps, a second hand holds the fiber at the opposite end of one of the clamps. This process is repeated if the fiber is not straight enough when attempting to align with it. This can take a significant amount of time, but is crucial to the fabrication of the LPGs since any small deviation can result in a failed LPG.

    \subsection*{Laser power and alignment conditions}

        While inscribing LPGs the alignment conditions of the laser were not perfect but close to optimal conditions. However, laser alignment does not appear to offer much importance other than modifying the inscription power.

        This is comes from the fact that during the inscription process the shape of the index modulation are straight and long lines that extend to the cladding. This is done by using a higher intensity  $I=[0.35,0.4]$ and setting a high Z-offset, so that the modulations caused by the Kerr  self-focusing effect form a single long modulation at the core of the fiber [\hyperref[fig:cross-sec]{Figure \ref{fig:cross-sec}}].
        
         \begin{figure}[ht!]
            \centering
            \includegraphics[width=0.5\textwidth]{./includes/cross-sec.jpg}
            \tikz[remember picture, overlay] \draw [red, thick] (-4.8,0.5)  -- ++ (-0.2,8) ;
            \tikz[remember picture, overlay] \draw [red, thick] (-4,0.5)  -- ++ (-0.2,8) ;
            \tikz[remember picture, overlay] \draw [red, thick, latex-latex] (-5.2,8.5)  -- (-4.4,8.5) ;
            \tikz[remember picture, overlay] \node [red] at (-5,8.8) {Core};
            \tikz[remember picture, overlay] \draw [latex-latex, black] (-7.5,9)  -- (-7.5,8) -- (-6.5,8) ;
            \tikz[remember picture, overlay] \node [black] at (-7.8,8.7) {x};
            \tikz[remember picture, overlay] \node [black] at (-6.7,7.8) {z};
            \tikz [remember picture, overlay] \draw [black, thick] (-8.1,7.8) circle (0.2) ++ (45:0.2) -- ++ (-135:0.4) ++ (45:0.2) ++ (135:0.2) -- ++ (-45:0.4);
            \tikz [remember picture, overlay] \node[black] at (-8.6,7.9) {y};
             \caption{Section of the modulated index of the LPG}
           \label{fig:cross-sec}
        \end{figure}
      
       
        The shaping of these modulations has not been further investigated and I recognize that there is much room for improvement. 

        Reducing the sub-period of the LPG is also a process that has not been thoroughly investigated, but I fear this may increase the transmission losses since the dispersion of light in the core of the fiber will be increased.

        \subsection*{fabrication method of the LPGs}

        

        \subsection*{Recipe for the fabrication of LPGs}

        \subsection*{Graphs}
      %  \pgfplotsset{try min ticks=5}

%:         \begin{figure*}[ht!]
%:           \centering
%:           \begin{tikzpicture}
%:               \begin{axis}[
%:                       width=0.5\textwidth, height=0.35\textwidth,
%:                       xmode=log, log x ticks with fixed point,
%:                       grid=both,
%:                       ymin=-50, ymax=-30,
%:                       xmin=1470, xmax=1600,
%:                       xlabel= Wavelength(nm), ylabel= Transmission(dB)]
%:                   
%:                   \addplot[blue, smooth] table[x=WL, y=Det3_IL, col sep=semicolon] {./data/540-1605.csv};
%:               \end{axis}
%:        \end{tikzpicture}
%:       \end{figure*}
%:
%:
%:       
%:       \begin{figure*}[ht!]
%:        \begin{floatrow}
%:            \ffigbox{
%:                \begin{tikzpicture}
%:                    \begin{axis}[
%:                            width=0.5\textwidth, height=0.35\textwidth,
%:                            xmode=log, log x ticks with fixed point,
%:                            grid=both,
%:                            ymin=-35, ymax=-15, 
%:                            xmin=1470, xmax=1600,
%:                            xlabel= Wavelength(nm), ylabel= Transmission(dB)]
%:                        
%:                        \addplot[blue, smooth] table[x=WL, y=Det3_IL, col sep=semicolon] {./data/340-155.csv};
%:                    \end{axis}
%:                \end{tikzpicture}
%:                
%:            }
%:       
%:            \ffigbox{
%:                \begin{tikzpicture}
%:                    \begin{axis}[
%:                            width=0.5\textwidth, height=0.35\textwidth,
%:                            xmode=log, log x ticks with fixed point,
%:                            grid=both,
%:                            ymin=-62,ymax=-37,
%:                            xmin=1470, xmax=1600,
%:                            xlabel= Wavelength(nm), ylabel= Transmission(dB)]
%:                        
%:                        \addplot[blue, smooth] table[x=WL, y=Det3_IL, col sep=semicolon] {./data/340-1550.csv};
%:                    \end{axis}
%:                \end{tikzpicture}
%:                
%:            }
%:       
%:        \end{floatrow}
%:       \end{figure*}
%:
%:
%:       
%:       \begin{figure*}[ht!]
%:        \begin{floatrow}
%:            \ffigbox{
%:                \begin{tikzpicture}
%:                    \begin{axis}[
%:                            width=0.5\textwidth, height=0.35\textwidth,
%:                            xmode=log, log x ticks with fixed point,
%:                            grid=both,
%:                            ymin=-50, ymax=-30,
%:                            xmin=1470, xmax=1600,
%:                            xlabel= Wavelength(nm), ylabel= Transmission(dB)]
%:                        
%:                        \addplot[blue, smooth] table[x=WL, y=Det4_IL, col sep=semicolon] {./data/540-1480.csv};
%:                    \end{axis}
%:                \end{tikzpicture}
%:                
%:            }
%:       
%:            \ffigbox{
%:                \begin{tikzpicture}
%:                    \begin{axis}[
%:                            width=0.5\textwidth, height=0.35\textwidth,
%:                            xmode=log, log x ticks with fixed point,
%:                            grid=both,
%:                            ymin=-40, ymax=-20,
%:                            xmin=1470, xmax=1600,
%:                            xlabel= Wavelength(nm), ylabel= Transmission(dB)]
%:                        
%:                        \addplot[blue, smooth] table[x=WL, y=Det3_IL, col sep=semicolon] {./data/540-1567-1.csv};
%:                    \end{axis}
%:                \end{tikzpicture}
%:                
%:            }
%:       
%:        \end{floatrow}
%:       \end{figure*}
%:       

       

\end{document}