
\documentclass{article}

\usepackage[utf8]{inputenc} 
\usepackage{amsmath}
\usepackage{amsfonts}
\usepackage{graphicx}

\title{Travail Experimental}
\author{The Author}


\begin{document}
\maketitle
\section{Estimation approximative de la focale d'une lentille mince convergente}
On place sur le sol à la verticale d'un plafonnier une feuille de papier blanc, en ajustant ensuite la position de la lentille par rapport au sol de façon a fomer sur la feuille une image nette, on mesure alors de façon précise la hauteur $h$ de la lentille.

On retient : $h=f'=22.2 \pm 0.5 cm$
\section{Estimation de la focale de la lentille à l'aide de la relation de Descartes}
On place en odre su la banc optique les supports suivant:
\begin{itemize}
	\item Une source lumineuse équppée en sortie d'un verre dépoli sur lequel est gravé une lettre d'alphabet qui tiendra lieu d'objet $[A]$
	\item La lentille convergente dont on a jusqu'ici mesuré grossièrement la focale image $f'$
	\item Un écran qui matérialisera l'image de l'objet donnée par la lentille.
\end{itemize}

On prend alors la distance $D=\overline{AA'}$ entre l'objet et l'écran comme étant la valeur du produit $4,5\times h$: $D=110 \pm 2.5cm$ où $\Delta D=2.5cm$\\

On lit sur le banc optique la position du pied de support de la lentille lorsqu'elle se trouove dans la position $1$

Après avoir placé l'écran à un distance $D'$, on fait varier la position de notre lentille sur le banc optique de façon a voir une image nette de l'onjet sur l'ecran, on relève alors un segment
de longueur $Delta$


\end{document}
