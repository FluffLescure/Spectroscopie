\usepackage[a4paper, total={16cm, 22.125cm}]{geometry}

% Page Style
\usepackage[]{environ}
% Cover Page 
\usepackage{tikz}
\makeatletter
\def\parsecomma#1,#2\endparsecomma{\def\page@x{#1}\def\page@y{#2}}
\tikzdeclarecoordinatesystem{page}{
    \parsecomma#1\endparsecomma
    \pgfpointanchor{current page}{north east}
    % Save the upper right corner
    \pgf@xc=\pgf@x%
    \pgf@yc=\pgf@y%
    % save the lower left corner
    \pgfpointanchor{current page}{south west}
    \pgf@xb=\pgf@x%
    \pgf@yb=\pgf@y%
    % Transform to the correct placement
    \pgfmathparse{(\pgf@xc-\pgf@xb)/2.*\page@x+(\pgf@xc+\pgf@xb)/2.}
    \expandafter\pgf@x\expandafter=\pgfmathresult pt
    \pgfmathparse{(\pgf@yc-\pgf@yb)/2.*\page@y+(\pgf@yc+\pgf@yb)/2.}
    \expandafter\pgf@y\expandafter=\pgfmathresult pt
}
\makeatother


% Object formatting
\usepackage[12pt]{moresize}
\usepackage[]{anyfontsize}
\usepackage{titlesec}
\usepackage{import}
\usepackage{floatrow}
\usepackage{enumitem}
\usepackage{changepage}
\usepackage[normalem]{ulem}
\usepackage{array}
\newcommand{\ul}[1]{\underline{#1}}

\usepackage[]{chngcntr}
\usepackage{ifthen}
\ifthenelse{\figcountdepth > 1}
  {\counterwithin{figure}{section}\counterwithin{table}{section}}
  {}

\usepackage[format=plain, labelfont=it, textfont=it]{caption}
\makeatletter
\def\@makecaption#1#2{%
    \vskip\abovecaptionskip
    \sbox\@tempboxa{\textit{#1.} #2}

       
   

    \ifdim \wd\@tempboxa >\hsize
        #1. #2\par
    \else
        \global \@minipagefalse
        \hb@xt@\hsize{\hfil\box\@tempboxa\hfil}
    \fi
    \vskip\belowcaptionskip}
\makeatother

\DeclareCaptionFormat{underline}{\uline{#1#2#3}\par}

% Sections
\titleformat{\section}{\fontsize{16}{19.2}\bfseries}{\thesection.}{0.25em}{}
\titleformat{\subsection}{\fontsize{14}{16.8}\bfseries}{\tab\thesubsection.}{0.25em}{}
\titleformat{\subsubsection}{\fontsize{10}{12}}{\uline{\thesubsubsection)\enspace}}{0em}{\uline}





% Geometry

% Typewritting

\setlength{\parskip}{1em}
\setlength{\parindent}{0em}


\newenvironment{items}[3][0pt]
{\def\closesep{#3}
    \vspace{#2}
    \begin{itemize}
        \setlength{\itemsep}{#1}
        \setlength{\topsep}{0pt}
        \setlength{\partopsep}{0pt}}
        {\end{itemize}
    \vspace{\closesep}}

\newenvironment{enum}[3][0pt]
{\defclosesep{#3}
    \vspace{#2}
    \begin{enumerate}
        \setlength{\itemsep}{#1}
        \setlength{\topsep}{0pt}
        \setlength{\partopsep}{0pt}}
        {\end{enumerate}
    \vspace{\closesep}}

\newenvironment{eq}[2]
{\def\closesep{#2}
    \vspace{#1}
    \begin{align*}}
        {\end{align*}
    \vspace{\closesep}}

\newenvironment{lfeq}[2]
{\def\closesep{#2}
    \vspace{#1}
    \begin{flalign*}}
        {\end{flalign*}
    \vspace{\closesep}}
% List Formatting


\NewEnviron{dent}[1]{
    \vspace{-10pt}
    \begin{adjustwidth}{7mm}{}
        \uline{#1}\hspace{2mm}
        \BODY
    \end{adjustwidth}
    \vspace{-10pt}
}


\usepackage[framemethod=tikz]{mdframed}
\newcounter{count_theorem}[section]\setcounter{count_theorem}{0}
\newcommand{\thetheorem}{\arabic{count_theorem}}

\newcounter{count_exercise}[section]\setcounter{count_exercise}{0}
\newcommand{\theexercise}{\arabic{count_exercise}}


\newenvironment{theorem}[1][]{
    \refstepcounter{count_theorem}
    \mdfsetup{
        linecolor=red!30,
        innerbottommargin=10pt,
        linewidth=2pt,
        topline=false,
        bottomline=false,
        rightline=false,
        shadow=true,
        shadowsize=4.5pt,
        frametitlerule=false,
        apptotikzsetting={
                \tikzset{
                    mdfbackground/.append style={
                            left color=red!8,right color=red!3
                        }
                }
            }
    }
    \begin{mdframed}[]\relax
        \ifstrempty{#1}
        {\textbf{Theorem~\thetheorem.} }
        {\textbf{Theorem~\thetheorem.~#1} }
        }
        {\end{mdframed}\vspace{-10pt}
}

\newenvironment{note}{
    \mdfsetup{innertopmargin=5pt,
        linecolor=gray!30,
        linewidth=2pt,
        topline=false,
        bottomline=false,
        rightline=false,
        frametitleaboveskip=0pt,
        shadow=false,
        shadowsize=4pt,
        frametitlerule=false,
        apptotikzsetting={
                \tikzset{
                    mdfbackground/.append style={
                            left color=gray!8,right color=gray!3
                        }
                }
            }
    }
    \begin{mdframed}[]\relax
        \textbf{Note. }
        }
        {\end{mdframed}\vspace{-10pt}
}

\newenvironment{example}{
    \mdfsetup{innertopmargin=5pt,
        linecolor=green!30,
        linewidth=2pt,
        topline=false,
        bottomline=false,
        rightline=false,
        frametitleaboveskip=0pt,
        shadow=false,
        shadowsize=4pt,
        frametitlerule=false,
        apptotikzsetting={
                \tikzset{
                    mdfbackground/.append style={
                            left color=green!7,right color=green!2
                        },
                    mdfframetitlebackground/.append style={
                            left color=green!7,right color=green!2
                        }
                }
            }
    }
    \begin{mdframed}[]\relax
        \textbf{Example. }
        }
        {\end{mdframed}\vspace{-10pt}
}


\usetikzlibrary{calc,arrows}

\tikzset{
    excursus arrow/.style={%
            line width=2pt,
            draw=gray!40,
            rounded corners=2ex,
        },
    excursus head/.style={
            fill=white,
            font=\bfseries\sffamily,
            text=gray!80,
            anchor=base west,
        },
    excursus line/.style={%
            line width=2pt,
            draw=gray!40,
            rounded corners=2ex,
        }
}

\newenvironment{exercise}[1][]{%
    \refstepcounter{count_exercise}
    \mdfsetup{
        singleextra={
                \path let \p1=(P), \p2=(O) in (\x2,\y1) coordinate (Q);
                \path let \p1=(Q), \p2=(O) in (\x1,{(\y1-\y2)/2}) coordinate (M);
                \path [excursus line] ($(O)+(5em,0ex)$) -| (M) |- ($(Q)+(20em,0ex)$);
                \node [excursus head] at ($(Q)+(2.5em,-0.75pt)$) {\ifstrempty{#1}{Exercise \theexercise}{Exercise \theexercise:~#1}};},
        firstextra={
                \path let \p1=(P), \p2=(O) in (\x2,\y1) coordinate (Q);
                \path [excursus arrow,-to] (O) |- ($(Q)+(12em,0ex)$) .. controls +(0:16em) and +(185:6em) .. ++(23em,2ex);},
        middlelinewidth=2.5em,middlelinecolor=white,
        hidealllines=true,topline=true,
        innertopmargin=0.5ex,
        innerbottommargin=2.5ex,
        innerrightmargin=2pt,
        innerleftmargin=2ex,
        skipabove=0.87\baselineskip,
        skipbelow=0.62\baselineskip,
    }
    \begin{mdframed}[]\relax}
        {\end{mdframed}\vspace{-10pt}
}

% Functions and Data Plotting
\usepackage{subfig,wrapfig,adjustbox,multirow}


% Plotting Style
\usepackage{graphicx,pgfplots}
\usetikzlibrary{arrows}
\usetikzlibrary {patterns,patterns.meta}
\usepgfplotslibrary{fillbetween}
\pgfplotsset{compat=1.18}

\usepgfplotslibrary{units}
% Logarithmic Scale
\pgfplotsset{
    log x ticks with fixed point/.style={
            xticklabel={
                    \pgfkeys{/pgf/fpu=true}
                    \pgfmathparse{exp(\tick)}%
                    \pgfmathprintnumber[fixed relative, precision=3]{\pgfmathresult}
                    \pgfkeys{/pgf/fpu=false}
                }
        }
}

